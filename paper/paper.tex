\documentclass[twoside]{wiss}

%\usepackage{ascmac}
%\usepackage[dvips]{graphicx}

\journalhead{EpisoPass: Holy Grail}

\begin{document}

\title{EpisoPass: エピソード記憶にもとづくパスワード生成}
%\etitle{EpisoPass: The Ultimate Passwords Management System}
\etitle{EpisoPass: Password Management based on Episodic Memories}

\author{増井俊之\affil{Toshiyuki Masui, 慶應義塾大学 環境情報学部}}

\begin{abstract}
忘却しにくいエピソード記憶にもとづく秘密の質問を使って
強力なパスワードを生成/管理するシステム「EpisoPass」を提案する。
EpisoPassは、
ユーザの作成した秘密の質問への回答にもとづいて
シード文字列を換字することによって強力なパスワードを生成する。
シード文字列や回答のバリエーションにより異なるパスワードが生成されるので
様々なサービスに対して異なるパスワードを生成できる。

秘密の質問の数が充分であれば、
あらゆる情報を公開しても強力なパスワードを生成して利用することができる。 (強い主張すぎるかも)
\end{abstract}

\maketitle

\section{はじめに}

個人認証のためにパスワードが現在広く利用されている。
パスワード認証には多くの問題があることが知られているが\cite{増井_ユニマガ}、
今後も長期にわたって利用され続けることが予想されるため、
適切に運用するための工夫が必要になっている。

パスワードの長期的記憶が難しいことはパスワード認証の大きな問題点のひとつである。
安全に運用するためには
パスワードはランダムで長い文字列であることが望ましいが、
そのようなものを頭の中に記憶しておくことは難しい。
また複数のサービスを利用する場合、
サービスごとに異なるパスワードを利用することが望ましいが、
すべてのパスワードを記憶しておくことはほとんど不可能である。
%
Flor\^{e}ncioの...における調査では...\cite{Florencio:2007:LSW:1242572.1242661}、
また2011年の野村総研の調査によれば、
一般的なユーザがパスワード認証を行なうサイトは平均19.4個で、
利用しているパスワードは平均3.1個であった\cite{野村総研}。
多数のパスワードを記憶することが困難であるため、
多くのユーザが同じパスワードを複数サイトで使い回しているのだと思われる。

異なるパスワードをすべて記憶することは不可能なのでどこかに記録しておく必要があるが、
パスワードをそのまま記録するのは危険なので、
複数のパスワードを秘密情報として扱うためのパスワード管理システムが利用されている。
パスワード管理システムは
1個の「マスターパスワード」を利用してあらゆるパスワードを管理するもので、
暗号化されたデータベースにパスワードを格納するもの%
\cite{OnePassword}%
\cite{Dashlane}%
\cite{ミルパス}%
\cite{LastPass}%
\cite{KeyPass}%
\cite{NortonIDSafe}%
\cite{IDManager}%
が多いが、マスターパスワードをサービス名で変換することによって
複数のパスワードを生成するシステム\cite{SuperGenPass}もある。
% 前者はデータベースを解読される危険があるのに対し、
% 後者はパスワード本体をどこにも保存していないためより安全であるが、
両者ともにマスターパスワードの記憶は必須であり、
マスターパスワードを盗まれたり忘れたりする危険がある。
% パスワードを毎日使わない人なら忘れてしまう可能性は高い。

ユーザはパスワードを忘れてしまうことが非常に多いため、
多くのサービスにおいて
パスワードを復元したり初期化したりする手段が用意されている。
ユーザが秘密の質問に対する答を登録し、
質問に正しく回答することによってパスワードを復元したりリセットできるものがある。
また、秘密の質問に答えることによって
パスワード管理システムのマスターパスワードを復元する
システム\cite{平野亮:2011-11-07}も提案されている。

新しく覚えた情報や新しく考えた情報はどうしても忘れてしまう可能性があるので、
新しく作成したパスワード文字列を記憶して認証に利用することは本質的に無理がある。
一方、既知で忘れることがないエピソード記憶を秘密の質問として
認証のために直接利用することができれば、
認証に必要な情報を忘れてしまうことがないはずである。
多くの画像認証システム\cite{小池英樹:2006-05-15}は
秘密の質問に対して正答を示すことによって認証を行なっているため
パスワードのようなものを記憶する必要がない。
%認証方法を忘れにくいという特長がある。
% 秘密の質問がパスワードと同じぐらい強力であれば、
% それをメインの認証手段にしてしまえばいいことになる。
%
画像認証システムはまだ普及していないためどこでも利用することができないが、
忘れないエピソード記憶を利用した秘密の質問への回答を
強力なパスワードに変換するシステムがあれば、
通常のパスワード認証を用いた現在の様々なサービス上で
パスワードを忘れる心配なく安全な認証ができるようになる。
本論文ではこのようなシステム「EpisoPass」について述べる。

\section{EpisoPass}

\subsection{EpisoPassの原理}

EpisoPassは、
ユーザが忘れることがない個人的なエピソード記憶を文字列に変換することによって
安全なパスワードを生成するシステムである。
パスワード文字列は以下の手順で生成される。

\begin{enumerate}
\item パスワード生成の「種」となる文字列を用意する。
以下ではこれを「シード文字列」と表現する。
\item 忘れることがない個人的なエピソード記憶にもとづく秘密の質問を複数作成し、
それぞれについてひとつの正答と複数の偽答を用意する。
\item 質問と回答の組にもとづいてシード文字列に換字操作を行なう。
すべてに正しく回答したとき生成される文字列をパスワードとして利用する。
\end{enumerate}

以下にEpisoPassの利用例を示す。

\paragraph{利用例1}

\paragraph{利用例2}

\subsection{パスワードの計算方法}

\subsection{画像の利用}

画像認証がイイと言ってる人も多い

画像も使えるよ

\section{考察}

\subsection{他者が答を知っているときの危険}

自転車のカギ番号を共有しているとき
それをEpisoPassで管理すると全質問の答がわかってしまう

家族のなぞなぞ問題を作るといった工夫

\subsection{運用レベル}

何を秘密にするかのレベルがある

\paragraph{EpisoPassの利用を秘密にする}

\begin{itemize}
\item EpisoPassの利用を秘密にする

これは相当安全なはず。
EpisoPassはあらゆる文字列を生成できるから。

\item シード文字列を秘密にする

SuperGenPassと同じレベル。
秘密の問題としてサービス名を使えば良い

\item すべて公開する

実はこれでも大丈夫。
すべての計算結果を他人に見せなければ総当たり攻撃は回避できる。

\end{itemize}

\subsection{既存パスワードの利用}

SuperGenPass\cite{SuperGenPass}は良いけれども
既存のパスワードを登録することができない。
EpisoPassは大丈夫である。

\subsection{エピソード記憶をいろんなものに変換する面白さ}

\subsection{指紋をパスワードに変換することなども考えられる}

指紋はコピーされる
脳はコピーされない

\subsection{安全性}

エントロピー

\subsection{秘密の質問の脆弱性}

秘密の質問を解くことによってパスワードをリセットできる機能があれば
パスワードを忘れても大丈夫だという利点はあるものの、
このような方式がシステムを脆弱にしていることが近年問題になっている\footnote{
  ..... どこかの記者
}。

簡単な質問でリセットできるのが問題になっている。

秘密の質問は実は忘れることが多いという問題があるらしい。

平野\cite{平野亮:2011-11-07}は
秘密の質問からマスターパスワードを復元する方法を提案している。
Shamirのなんたら方式を使い、全部正解しなくても復元できるのがミソ。

パスワードは自分で考えることを想定しているのかもしれない?

\textbf{どれが正解か知っていなければこの方法は適用できないのでは?}


EpisoPassは、
パスワードを忘れたときに秘密の質問を利用するのではなく、
パスワードの生成に秘密の質問を利用するという点が異なっている。

兼子は、
秘密の質問のエントロピを上げる方法を提案している\cite{Kaneko}。

多くのシステムにおいて、
「母親の旧姓は?」や「最初に飼ったペットの名前は?」のような、
あらかじめ決まった質問のみを利用できるようになっているが、
このような問題は他人が解くことも容易であるうえに、
秘密の質問の数は少ないのが普通だからである\cite{Rabkin:2008:PKQ:1408664.1408667}。
% 銀行20個調べて秘密の質問の弱さを指摘

EpisoPassはシステムに用意された問題を利用せず、
ユーザが自分で考えたなぞなぞ問題を利用するので、
他人には解くことが難しく自分では忘れない問題を自由に作成できる。
これは強力なはずであるが、
そのような問題を作成することは難しいということが
知られている\cite{Just:2009:PCC:1572532.1572543}\cite{Schechter:2009:NSM:1607723.1608145}。
%
%  秘密の質問の問題
%   自分で[[秘密の質問]]を決めるととても弱くなることが判明
%   複数の秘密の質問を使うとマシになる?
%   結構人は間違うものらしい
%  applicability or repeatability の問題がある
%   Applicability: How widely applicable is the given question?
%   Memorability: How easy is it for the user to recall the answer?
%   Repeatability: How accurately can the answer be replayed, without syntactic or semantic ambiguity?
%  ユーザに質問を作らせると全然駄目なことが多い
%   すぐ解けてしまうもの
%   思い出せないもの
%
% 自分が作るものでくだらない秘密の質問は駄目
%
古い記憶にもとづいて作成した秘密の問題はユーザが想像するよりも解かれやすい。
この問題を避けるため、問題と答を連想するために画像を利用したり、
複数の問題を連続的に利用したりする手法も提案されている\cite{Renaud:2010:PQE:2146303.2146318}。

%  昔の記憶にもとづく[[秘密の質問]]はアタックされやすい
%  質問に写真を加えることにより思い出しやすくなったり
%  解決策
%   答を思い出すために画像を使う
%    動物や場所の写真から人を思い出すなど
%   必ずしも本当でない連想を使う
%   連続的に出題する

% 秘密の質問はシステムに指定される場合とユーザが作成できる場合がある。
% システムが質問を提供する場合、あらゆるユーザが答えられる質問にしなければならないため、
% 「ペットの名前は?」「母親の旧姓は?」
% のような簡単なものであることが多いが、
% このような質問は他人が解くことも容易であるうえに、
% 質問の数も充分でないことが多いため充分安全でない\cite{Rabkin:2008:PKQ:1408664.1408667}。
% %
% ユーザが質問を作成できる場合、
% 他人に解くことが難しく自分では忘れない問題を自由に作成できるはずであるが、
% そのような問題を作成することは難しいということが
% 知られている\cite{Just:2009:PCC:1572532.1572543}\cite{Schechter:2009:NSM:1607723.1608145}。

\bibliographystyle{jwiss}
\bibliography{paper}

\end{document}
